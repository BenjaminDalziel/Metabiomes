\documentclass{article}

% Packages for applied mathematics
\usepackage{amsmath} % Enhanced math functionality
\usepackage{amsfonts} % Additional math fonts
\usepackage{amssymb} % Additional math symbols
\usepackage{parskip}

\begin{document}

\section*{Supplemental Material}

\subsection*{Generalized Lotka-Voltera}

The generalized Lotka-Volterra model is
\begin{align}
\dfrac{dN_i}{dt} &= N_i\left(r_i + \sum_k a_{ik}N_k\right) \equiv f_i(N)\\
&=r_iN_i + N_i\sum_k a_{ik}N_k \\
&=r_iN_i + a_{ii} N_i^2 + N_i \sum_{k\neq i} a_{ik}N_k
\end{align}

Assume there is an equilibrium $N^*$ where $N^*_i>0$. For all $i$ that equilibrium must satisfy
\begin{equation}
f_i(N^*) = 0
\end{equation}

\begin{equation}
r_iN^*_i + a_{ii} (N^*_i)^2 + N^*_i \sum_{k\neq i} a_{ik}N^*_k = 0
\end{equation}

\begin{equation}
r_i + a_{ii} N^*_i + \sum_{k\neq i} a_{ik}N^*_k = 0 
\end{equation}

So 
\begin{equation}
-a_{ii} N^*_i = r_i + \sum_{k\neq i} a_{ik}N^*_k
\end{equation}

The diagonal of the Jacobian is
\begin{equation}
\dfrac{\partial f_i}{\partial N_i} = r_i + 2a_{ii} N_i + \sum_{k\neq i} a_{ik}N_k
\end{equation}

We evaluate the Jacobian at the equilrium $N_i^*$, using the relationship given in equation 7
\begin{align}
    \left.\dfrac{\partial f_i}{\partial N_i} \right \rvert_{N^*} 
    &= r_i + 2a_{ii} N^*_i + \sum_{k\neq i} a_{ik}N^*_k \\
    &= 2a_{ii} N^*_i - a_{ii}N^*_i \\ 
    &= a_{ii} N^*_i
\end{align}


\subsection*{Generalized Lotka-Volterra in a Metacommunity}

Let $N_i$ represent the population size for a particular taxa-location combination within the metacommunity. Let $t(i)$ return the taxa associated with $i$ and let $l(i)$ return the location associated with $i$.

The generalized Lotka-Volterra model is
\begin{align}
\dfrac{dN_i}{dt} 
&= N_i\left(r_i+\sum_k a_{ik}N_k\right)-mN_i+\sum_k m_{ik}N_k \equiv f_i(N) \\
\notag\\
&=r_iN_i + a_{ii}N_i^2 + N_i\sum_{k\neq i} a_{ik}N_k - mN_i + \sum_k m_{ik} N_k
\end{align}
where $a_{ii} = s_i$ represents the strength of intraspecific competition.

Species interactions only occur for different taxa in the same location. Thus
\begin{equation}
    a_{ij} = 
    \left\{
        \begin{array}{ll}
            X_{ij}, & \text{if } l(i) = l(j) \text{ and } t(i) \neq t(j)\\
            \\
            0, & \text{otherwise}
            \end{array}
            \right.
\end{equation}
where $X_{ij}$ is a random variable whose distribution is parameterized as described below.

Individuals emigrate from a location at per-capita rate $m$. We assume that individual who emigrate select another location to immigrate to at random, with each of the $M-1$ other locations are given equal weight. Thus the rate at which individuals emigrating from $j$ arrive at $i$ is given by
\begin{equation}
    m_{ij} = 
    \left\{
        \begin{array}{ll}
            \dfrac{m}{M-1}, & \text{if } l(i) \neq l(j) \text{ and } t(i) = t(j) \\
            \\
            0, & \text{otherwise}
        \end{array}
        \right.
\end{equation}
where the 0s occur because migration can only occur when $i$ and $j$ refer to the same taxa (in different locations).

\subsection*{Jacobian}
For the diagonal,
\begin{equation}
\dfrac{\partial f_i}{\partial N_i} = r_i + 2a_{ii}N_i + \sum_{k \neq i} a_{ik}N_k - m 
\end{equation}

For the off-diagonal,
\begin{equation}
\dfrac{\partial f_i}{\partial N_j} = a_{ij} N_i + m_{ij}
\end{equation}

Evaluating the Jacobian at the equilibrium $N^*$. Assume there is an equilibrium $N^*$ that satisfies $N^*_i>0$ for all $i$.
\begin{equation}
    N^*_i\left(r_i+\sum_k a_{ik}N^*_k\right)-mN^*_i+\sum_k m_{ik}N_k=0
\end{equation}
Dividing by $N_i^*$
\begin{equation}
r_i+\sum_k a_{ik}N^*_k-m+\dfrac{1}{N^*_i}\sum_k m_{ik}N^*_k=0
\end{equation}

Breaking apart the first sum to pull out intraspecific competition,
\begin{equation}
    r_i+a_{ii}N_i^* + \sum_{k \neq i} a_{ik}N^*_k-m+\dfrac{1}{N^*_i}\sum_k m_{ik}N^*_k=0
\end{equation}
So
\begin{equation}
    r_i+\sum_{k \neq i} a_{ik}N^*_k-m=-a_{ii}N_i^* - \dfrac{1}{N^*_i}\sum_k m_{ik}N^*_k
\end{equation}
So
\begin{align}
    \left.\dfrac{\partial f_i}{\partial N_i} \right \rvert_{N^*} 
    &= r_i + 2a_{ii} N^*_i + \sum_{k\neq i} a_{ik}N^*_k - m \\
    &= 2a_{ii}N^*_i-a_{ii}N_i^* - \dfrac{1}{N^*_i}\sum_k m_{ik}N^*_k \\
    &= a_{ii}N^*_i- \dfrac{1}{N^*_i}\sum_k m_{ik}N^*_k
\end{align}
and
\begin{align}
    \left.\dfrac{\partial f_i}{\partial N_j} \right \rvert_{N^*} 
    &= a_{ij} N^*_i + m_{ij}
\end{align}

\end{document}